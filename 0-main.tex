%----------------------------------------------------------------------------------
% Exemplo do uso da classe pucrs-ppgcc.cls.
%----------------------------------------------------------------------------------

% Seleção de idioma da monografia. Por enquanto as únicas opções
% suportadas são 'portuguese' e 'english'
% Para impressão em frente e verso, use a opção 'twoside'. Da
% mesma forma, use 'oneside' para impressão em um lado apenas.
\documentclass[english,oneside]{pucrs-ppgcc}
%\documentclass[english,twoside]{pucrs-ppgcc}

%----------------------------------------------------------------
% Coloque seus pacotes abaixo.
%
% Obs.: muitos pacotes de uso comum do LaTeX, como amsmath,
% geometry e url já são automaticamente incluídos pela classe
% (veja o arquivo .cls). Isso torna obrigatória a presença destes
% no sistema para o uso desta classe, mas ao mesmo tempo o uso se
% torna mais simples.  Recomendo a instalação da versão mais
% recente da distribuição TeXLive (para Windows e UNIXes):
% www.tug.org/texlive/
%
% Pacotes e opções já incluídas automaticamente:
%
% \RequirePackage[T1]{fontenc}[2005/09/27]
% \RequirePackage[utf8x]{inputenc}[2008/03/30]
% \RequirePackage[english,brazil]{babel}[2008/07/06]
% \RequirePackage[a4paper]{geometry}[2010/09/12]
% \RequirePackage{textcomp}[2005/09/27]
% \RequirePackage{lmodern}[2009/10/30]
% \RequirePackage{indentfirst}[1995/11/23]
% \RequirePackage{setspace}[2000/12/01]
% \RequirePackage{textcase}[2004/10/07]
% \RequirePackage{float}[2001/11/08]
% \RequirePackage{amsmath}[2000/07/18]
% \RequirePackage{amssymb}[2009/06/22]
% \RequirePackage{amsfonts}[2009/06/22]
% \RequirePackage{url}
% \RequirePackage[table]{xcolor}[2007/01/21]
%\RequirePackage{array}[2008/09/09]
%\RequirePackage{longtable}
%----------------------------------------------------------------
% Para inserção de figuras.
\usepackage{graphicx}
% Utilize a opção 'pdftex' se você estiver usando o pdflatex (que
% permite figuras em formatos como .jpg ou .png)
%\usepackage[pdftex]{graphicx}

% Para tabelas com elementos ocupando mais de uma linha
\usepackage{multirow}
% Para frações na mesma linha (ex. ⅓).
\usepackage{nicefrac}
% Para inserir figuras lado a lado.
% \usepackage{subfigure}
% Para formatar algoritmos.
% A opção [algo2e] é necessária para evitar conflitos
% com as definições da classe.
%\usepackage[algo2e]{algorithm2e}
\usepackage{algorithmic}
% Um float do tipo algoritmo. No momento
% este pacote é incompatível com a classe.
%\usepackage{algorithm}
\usepackage{bookmark}

%----------------------------------------------------------------
% Autor (OBRIGATÓRIO)
%----------------------------------------------------------------
\author{Atila Leites Romero}

%----------------------------------------------------------------
% Título (OBRIGATÓRIO). Devem ser passados DOIS parâmetros,
% o título em português E o inglês, não importando o idioma
% escolhido. Os títulos são utilizados para a montagem da capa,
% resumo e abstract mais tarde.
%----------------------------------------------------------------
\title{Data-carving usando redes neurais BLSTM}
      {Data-carving using BLSTM neural networks}

%----------------------------------------------------------------
% Opções para o tipo de trabalho (OBRIGATÓRIO)
%----------------------------------------------------------------
%\tipotrabalho{\monografia}  % Monografias em geral (e de "bônus": TCCs)
\tipotrabalho{\pep}         % Plano de estudo e pesquisa
%\tipotrabalho{\dissertacao} % Dissertação
%\tipotrabalho{\ptese}       % Proposta de tese
%\tipotrabalho{\tese}        % Tese

%----------------------------------------------------------------
% Seleção do curso ("este trabalho é um requisito parcial para
% obtenção do grau de (mestre ou doutor) em Ciência da Computação").
% Necessário somente para o tipo 'monografia'.
%----------------------------------------------------------------
%\grau{\bacharel} % Este é "bônus"
\grau{\mestre}
%\grau{\doutor}

%----------------------------------------------------------------
% Orientador (e Co-orientador, caso haja um). É OBRIGATÓRIO
% informar pelo menos o orientador.
%----------------------------------------------------------------
\orientador{Avelino Francisco Zorzo}
%\coorientador{Ciclano de Farias}

%----------------------------------------------------------------
% A capa é inserida automaticamente. Por isso não é necessário
% chamar \maketitle
%----------------------------------------------------------------
\begin{document}

%----------------------------------------------------------------
% Depois da capa vem a dedicatória e a epígrafe.
%----------------------------------------------------------------
\dedicatoria{Dedico este trabalho à minha família.}

\epigrafe{If we are offered several hypotheses, we should begin our considerations by striking the most complex of them with our sword.}
         {Isaac Asimov}

%----------------------------------------------------------------
% Também dá para fazer as duas na mesma página:
%----------------------------------------------------------------
%\dedigrafe{Dedico este trabalho a meus pais.}
%          {The art of simplicity is a puzzle of complexity.}
%          {Douglas Horton}

%----------------------------------------------------------------
% A seguir, a página de agradecimentos (OPCIONAL):
%----------------------------------------------------------------
\begin{agradecimentos}
%O presente trabalho foi realizado com apoio da Coordenação de Aperfeiçoamento de Pessoal Nivel Superior – Brasil (CAPES) – Código de Financiamento 001
This study was financed in part by the Coordenação de Aperfeiçoamento de Pessoal de NivelSuperior – Brasil (CAPES) – Finance Code 001

CAPE PUC INCT e orientador
\end{agradecimentos}

%----------------------------------------------------------------
% Resumo, com as palavras-chave passadas por parâmetro
% (OBRIGATÓRIO, ao menos para teses e dissertações)
%----------------------------------------------------------------
\begin{resumo}{lorem, ipsum, dolor, sit, amet}
Seu resumo em português aqui. lorem ipsum dolor sit amet
consetetur sadipscing elitr sed diam nonumy eirmod tempor invidunt
ut labore et dolore magna aliquyam erat sed diam voluptua at vero
eos et accusam et justo duo dolores et ea rebum stet clita.  kasd
gubergren no sea takimata sanctus est lorem ipsum dolor sit amet
lorem ipsum dolor sit amet consetetur sadipscing elitr sed diam
nonumy eirmod tempor invidunt ut labore et dolore magna aliquyam
erat sed diam voluptua at.
\end{resumo}

%----------------------------------------------------------------
% Abstract, com as palavras-chave passadas por parâmetro
% (OBRIGATÓRIO, ao menos para teses e dissertações)
%----------------------------------------------------------------
\begin{abstract}{lorem, ipsum, dolor, sit, amet}
Your abstract in English here. lorem ipsum dolor sit amet
consetetur sadipscing elitr sed diam nonumy eirmod tempor invidunt
ut labore et dolore magna aliquyam erat sed diam voluptua at vero
eos et accusam et justo duo dolores et ea rebum stet clita kasd
gubergren no sea takimata sanctus est lorem ipsum dolor sit amet
lorem ipsum dolor sit amet consetetur sadipscing elitr sed diam
nonumy eirmod tempor invidunt ut labore et dolore magna aliquyam
erat sed diam voluptua at
\end{abstract}

%----------------------------------------------------------------
% Listas e sumário, nessa ordem. Somente o sumário é obrigatório,
% portanto, comente as outras listas, caso sejam desnecessárias.
%----------------------------------------------------------------
% \listoffigures       % Lista de figuras      (OPCIONAL)
% \listoftables        % Lista de tabelas      (OPCIONAL)
% \listofalgorithms    % Lista de algoritmos   (OPCIONAL)
% \listofacronyms      % Lista de siglas       (OPCIONAL)
% \listofabbreviations % Lista de abreviaturas (OPCIONAL)
% \listofsymbols       % Lista de símbolos     (OPCIONAL)
\tableofcontents     % Sumário               (OBRIGATÓRIO)

%----------------------------------------------------------------
% Aqui começa o desenvolvimento do trabalho. Para uma melhor
% organização do documento, separe-o em arquivos,
% um para cada capítulo. Para isso, utilize o comando \include,
% como mostrado abaixo.
%----------------------------------------------------------------
\chapter{Introduction}
%  motivacao
%  objetivos
%  organizacao
The remainder of this paper is organized as follows:
section 2 gives details on preliminary searches, explaining why the field of sequence learning was elected as the focus of this study, section 3 describes the search process and execution, section 4 discusses the results found and section 5 comments on the planning of further research based on the answered questions.


\chapter{File deletion and recovering}
% o que acontece quando um arquivo é deletado
%   Exemplo de arquivo em um sistema de arquivos NTFS
%     MFT
%       directory entry
%       file entry
%       entry pointing to position in disk
%         comment append, fragmentation and slack
%       file deletion
% recover strategies
%  using MFT data
%  find lost MFT data
%  data-carving

\chapter{Data-carving}
% desafios do data-carving
%  fragmentacao
%  reconhecimento de estruturas
% solucoes existentes
%  abordagens comuns
%   magic numbers
%   skip known files
%   use content to infer file size
%   use fixed value as file size
%   watch content to infer when file ended
%   test opening the file
% exemplo de arquivo 
Data carving is a forensic process that attempts to recover files without previous information of where the file starts or ends.
To accomplish this, a software has to analyze a source of raw data, searching for patterns indicating a know file type and making attempts to locate and reconstruct each of its constituent parts.
That process commonly disregards the filesystem, being able to recover deleted files from unallocated areas, but faces the problem of fragmentation: in many cases, files are not written sequentially on disk and deleted files may have missing parts.

For example, while doing data carving on a hard drive, a software could sequentially read each drive sector, find a known header of a video file and save the following sectors until a footer is found or a size limit is reached. This is a common data carving approach, but one that fails to recover fragmented videos.

The patterns searched by data carving software are generally manually coded, taking advantage of fixed byte sequences found on headers and footers. But the amount of different file types combined with the slow process of manually coding each of those patterns makes the development of data carving software a tedious task.

The application of machine learning solutions to this manual task has the potential to make it easier and faster. An initial strategy could be to train a classifier to, given a chunk of data, provide a label indicating a file type. That could be used to recover unfragmented deleted files.

The recovery of fragmented files through data carving would require some sort of pattern recognition on the identified chunks, in order to reconstruct the correct sequence.

This systematic mapping study objective is to identify some of the potential machine learning approaches that could be used to automate the construction of data carving software.

\chapter{Machine learning}

\chapter{Proposal}

\chapter{Schedule}




%----------------------------------------------------------------
% Aqui vai a bibliografia. Existem dois estilos de citação: use
% 'ppgcc-alpha' para citações do tipo [Abc+] ou [XYZ] (em ordem
% alfabética na bibliografia), e 'ppgcc-num' para citações
% numéricas do tipo [1], [20], etc., em ordem de referência.
%----------------------------------------------------------------
\bibliographystyle{ppgcc-alpha}
%\bibliographystyle{ppgcc-num}
%\bibliographystyle{apalike}
\bibliography{bib}

%----------------------------------------------------------------
% Após \appendix, se iniciam os capítulos de Apêndice, com
% numeração alfabética.
%----------------------------------------------------------------
% \appendix
% \chapter{Meu primeiro apêndice}
% \chapter{My second appendix}

%----------------------------------------------------------------
% Aqui vão os "capítulos" de anexos. Cada anexo deve
% ser considerado um capítulo.
%----------------------------------------------------------------
% \anexos
% \chapter{Meu primeiro anexo}
% \chapter{My second attachment}

% E aqui (para a felicidade de todos) termina o documento.
\end{document}
